
% Default to the notebook output style

    


% Inherit from the specified cell style.




    
\documentclass[11pt]{article}

    
    
    \usepackage[T1]{fontenc}
    % Nicer default font (+ math font) than Computer Modern for most use cases
    \usepackage{mathpazo}

    % Basic figure setup, for now with no caption control since it's done
    % automatically by Pandoc (which extracts ![](path) syntax from Markdown).
    \usepackage{graphicx}
    % We will generate all images so they have a width \maxwidth. This means
    % that they will get their normal width if they fit onto the page, but
    % are scaled down if they would overflow the margins.
    \makeatletter
    \def\maxwidth{\ifdim\Gin@nat@width>\linewidth\linewidth
    \else\Gin@nat@width\fi}
    \makeatother
    \let\Oldincludegraphics\includegraphics
    % Set max figure width to be 80% of text width, for now hardcoded.
    \renewcommand{\includegraphics}[1]{\Oldincludegraphics[width=.8\maxwidth]{#1}}
    % Ensure that by default, figures have no caption (until we provide a
    % proper Figure object with a Caption API and a way to capture that
    % in the conversion process - todo).
    \usepackage{caption}
    \DeclareCaptionLabelFormat{nolabel}{}
    \captionsetup{labelformat=nolabel}

    \usepackage{adjustbox} % Used to constrain images to a maximum size 
    \usepackage{xcolor} % Allow colors to be defined
    \usepackage{enumerate} % Needed for markdown enumerations to work
    \usepackage{geometry} % Used to adjust the document margins
    \usepackage{amsmath} % Equations
    \usepackage{amssymb} % Equations
    \usepackage{textcomp} % defines textquotesingle
    % Hack from http://tex.stackexchange.com/a/47451/13684:
    \AtBeginDocument{%
        \def\PYZsq{\textquotesingle}% Upright quotes in Pygmentized code
    }
    \usepackage{upquote} % Upright quotes for verbatim code
    \usepackage{eurosym} % defines \euro
    \usepackage[mathletters]{ucs} % Extended unicode (utf-8) support
    \usepackage[utf8x]{inputenc} % Allow utf-8 characters in the tex document
    \usepackage{fancyvrb} % verbatim replacement that allows latex
    \usepackage{grffile} % extends the file name processing of package graphics 
                         % to support a larger range 
    % The hyperref package gives us a pdf with properly built
    % internal navigation ('pdf bookmarks' for the table of contents,
    % internal cross-reference links, web links for URLs, etc.)
    \usepackage{hyperref}
    \usepackage{longtable} % longtable support required by pandoc >1.10
    \usepackage{booktabs}  % table support for pandoc > 1.12.2
    \usepackage[inline]{enumitem} % IRkernel/repr support (it uses the enumerate* environment)
    \usepackage[normalem]{ulem} % ulem is needed to support strikethroughs (\sout)
                                % normalem makes italics be italics, not underlines
    

    
    
    % Colors for the hyperref package
    \definecolor{urlcolor}{rgb}{0,.145,.698}
    \definecolor{linkcolor}{rgb}{.71,0.21,0.01}
    \definecolor{citecolor}{rgb}{.12,.54,.11}

    % ANSI colors
    \definecolor{ansi-black}{HTML}{3E424D}
    \definecolor{ansi-black-intense}{HTML}{282C36}
    \definecolor{ansi-red}{HTML}{E75C58}
    \definecolor{ansi-red-intense}{HTML}{B22B31}
    \definecolor{ansi-green}{HTML}{00A250}
    \definecolor{ansi-green-intense}{HTML}{007427}
    \definecolor{ansi-yellow}{HTML}{DDB62B}
    \definecolor{ansi-yellow-intense}{HTML}{B27D12}
    \definecolor{ansi-blue}{HTML}{208FFB}
    \definecolor{ansi-blue-intense}{HTML}{0065CA}
    \definecolor{ansi-magenta}{HTML}{D160C4}
    \definecolor{ansi-magenta-intense}{HTML}{A03196}
    \definecolor{ansi-cyan}{HTML}{60C6C8}
    \definecolor{ansi-cyan-intense}{HTML}{258F8F}
    \definecolor{ansi-white}{HTML}{C5C1B4}
    \definecolor{ansi-white-intense}{HTML}{A1A6B2}

    % commands and environments needed by pandoc snippets
    % extracted from the output of `pandoc -s`
    \providecommand{\tightlist}{%
      \setlength{\itemsep}{0pt}\setlength{\parskip}{0pt}}
    \DefineVerbatimEnvironment{Highlighting}{Verbatim}{commandchars=\\\{\}}
    % Add ',fontsize=\small' for more characters per line
    \newenvironment{Shaded}{}{}
    \newcommand{\KeywordTok}[1]{\textcolor[rgb]{0.00,0.44,0.13}{\textbf{{#1}}}}
    \newcommand{\DataTypeTok}[1]{\textcolor[rgb]{0.56,0.13,0.00}{{#1}}}
    \newcommand{\DecValTok}[1]{\textcolor[rgb]{0.25,0.63,0.44}{{#1}}}
    \newcommand{\BaseNTok}[1]{\textcolor[rgb]{0.25,0.63,0.44}{{#1}}}
    \newcommand{\FloatTok}[1]{\textcolor[rgb]{0.25,0.63,0.44}{{#1}}}
    \newcommand{\CharTok}[1]{\textcolor[rgb]{0.25,0.44,0.63}{{#1}}}
    \newcommand{\StringTok}[1]{\textcolor[rgb]{0.25,0.44,0.63}{{#1}}}
    \newcommand{\CommentTok}[1]{\textcolor[rgb]{0.38,0.63,0.69}{\textit{{#1}}}}
    \newcommand{\OtherTok}[1]{\textcolor[rgb]{0.00,0.44,0.13}{{#1}}}
    \newcommand{\AlertTok}[1]{\textcolor[rgb]{1.00,0.00,0.00}{\textbf{{#1}}}}
    \newcommand{\FunctionTok}[1]{\textcolor[rgb]{0.02,0.16,0.49}{{#1}}}
    \newcommand{\RegionMarkerTok}[1]{{#1}}
    \newcommand{\ErrorTok}[1]{\textcolor[rgb]{1.00,0.00,0.00}{\textbf{{#1}}}}
    \newcommand{\NormalTok}[1]{{#1}}
    
    % Additional commands for more recent versions of Pandoc
    \newcommand{\ConstantTok}[1]{\textcolor[rgb]{0.53,0.00,0.00}{{#1}}}
    \newcommand{\SpecialCharTok}[1]{\textcolor[rgb]{0.25,0.44,0.63}{{#1}}}
    \newcommand{\VerbatimStringTok}[1]{\textcolor[rgb]{0.25,0.44,0.63}{{#1}}}
    \newcommand{\SpecialStringTok}[1]{\textcolor[rgb]{0.73,0.40,0.53}{{#1}}}
    \newcommand{\ImportTok}[1]{{#1}}
    \newcommand{\DocumentationTok}[1]{\textcolor[rgb]{0.73,0.13,0.13}{\textit{{#1}}}}
    \newcommand{\AnnotationTok}[1]{\textcolor[rgb]{0.38,0.63,0.69}{\textbf{\textit{{#1}}}}}
    \newcommand{\CommentVarTok}[1]{\textcolor[rgb]{0.38,0.63,0.69}{\textbf{\textit{{#1}}}}}
    \newcommand{\VariableTok}[1]{\textcolor[rgb]{0.10,0.09,0.49}{{#1}}}
    \newcommand{\ControlFlowTok}[1]{\textcolor[rgb]{0.00,0.44,0.13}{\textbf{{#1}}}}
    \newcommand{\OperatorTok}[1]{\textcolor[rgb]{0.40,0.40,0.40}{{#1}}}
    \newcommand{\BuiltInTok}[1]{{#1}}
    \newcommand{\ExtensionTok}[1]{{#1}}
    \newcommand{\PreprocessorTok}[1]{\textcolor[rgb]{0.74,0.48,0.00}{{#1}}}
    \newcommand{\AttributeTok}[1]{\textcolor[rgb]{0.49,0.56,0.16}{{#1}}}
    \newcommand{\InformationTok}[1]{\textcolor[rgb]{0.38,0.63,0.69}{\textbf{\textit{{#1}}}}}
    \newcommand{\WarningTok}[1]{\textcolor[rgb]{0.38,0.63,0.69}{\textbf{\textit{{#1}}}}}
    
    
    % Define a nice break command that doesn't care if a line doesn't already
    % exist.
    \def\br{\hspace*{\fill} \\* }
    % Math Jax compatability definitions
    \def\gt{>}
    \def\lt{<}
    % Document parameters
    \title{DB0201EN-Week4-1-1-Analyzing-2-py}
    
    
    

    % Pygments definitions
    
\makeatletter
\def\PY@reset{\let\PY@it=\relax \let\PY@bf=\relax%
    \let\PY@ul=\relax \let\PY@tc=\relax%
    \let\PY@bc=\relax \let\PY@ff=\relax}
\def\PY@tok#1{\csname PY@tok@#1\endcsname}
\def\PY@toks#1+{\ifx\relax#1\empty\else%
    \PY@tok{#1}\expandafter\PY@toks\fi}
\def\PY@do#1{\PY@bc{\PY@tc{\PY@ul{%
    \PY@it{\PY@bf{\PY@ff{#1}}}}}}}
\def\PY#1#2{\PY@reset\PY@toks#1+\relax+\PY@do{#2}}

\expandafter\def\csname PY@tok@w\endcsname{\def\PY@tc##1{\textcolor[rgb]{0.73,0.73,0.73}{##1}}}
\expandafter\def\csname PY@tok@c\endcsname{\let\PY@it=\textit\def\PY@tc##1{\textcolor[rgb]{0.25,0.50,0.50}{##1}}}
\expandafter\def\csname PY@tok@cp\endcsname{\def\PY@tc##1{\textcolor[rgb]{0.74,0.48,0.00}{##1}}}
\expandafter\def\csname PY@tok@k\endcsname{\let\PY@bf=\textbf\def\PY@tc##1{\textcolor[rgb]{0.00,0.50,0.00}{##1}}}
\expandafter\def\csname PY@tok@kp\endcsname{\def\PY@tc##1{\textcolor[rgb]{0.00,0.50,0.00}{##1}}}
\expandafter\def\csname PY@tok@kt\endcsname{\def\PY@tc##1{\textcolor[rgb]{0.69,0.00,0.25}{##1}}}
\expandafter\def\csname PY@tok@o\endcsname{\def\PY@tc##1{\textcolor[rgb]{0.40,0.40,0.40}{##1}}}
\expandafter\def\csname PY@tok@ow\endcsname{\let\PY@bf=\textbf\def\PY@tc##1{\textcolor[rgb]{0.67,0.13,1.00}{##1}}}
\expandafter\def\csname PY@tok@nb\endcsname{\def\PY@tc##1{\textcolor[rgb]{0.00,0.50,0.00}{##1}}}
\expandafter\def\csname PY@tok@nf\endcsname{\def\PY@tc##1{\textcolor[rgb]{0.00,0.00,1.00}{##1}}}
\expandafter\def\csname PY@tok@nc\endcsname{\let\PY@bf=\textbf\def\PY@tc##1{\textcolor[rgb]{0.00,0.00,1.00}{##1}}}
\expandafter\def\csname PY@tok@nn\endcsname{\let\PY@bf=\textbf\def\PY@tc##1{\textcolor[rgb]{0.00,0.00,1.00}{##1}}}
\expandafter\def\csname PY@tok@ne\endcsname{\let\PY@bf=\textbf\def\PY@tc##1{\textcolor[rgb]{0.82,0.25,0.23}{##1}}}
\expandafter\def\csname PY@tok@nv\endcsname{\def\PY@tc##1{\textcolor[rgb]{0.10,0.09,0.49}{##1}}}
\expandafter\def\csname PY@tok@no\endcsname{\def\PY@tc##1{\textcolor[rgb]{0.53,0.00,0.00}{##1}}}
\expandafter\def\csname PY@tok@nl\endcsname{\def\PY@tc##1{\textcolor[rgb]{0.63,0.63,0.00}{##1}}}
\expandafter\def\csname PY@tok@ni\endcsname{\let\PY@bf=\textbf\def\PY@tc##1{\textcolor[rgb]{0.60,0.60,0.60}{##1}}}
\expandafter\def\csname PY@tok@na\endcsname{\def\PY@tc##1{\textcolor[rgb]{0.49,0.56,0.16}{##1}}}
\expandafter\def\csname PY@tok@nt\endcsname{\let\PY@bf=\textbf\def\PY@tc##1{\textcolor[rgb]{0.00,0.50,0.00}{##1}}}
\expandafter\def\csname PY@tok@nd\endcsname{\def\PY@tc##1{\textcolor[rgb]{0.67,0.13,1.00}{##1}}}
\expandafter\def\csname PY@tok@s\endcsname{\def\PY@tc##1{\textcolor[rgb]{0.73,0.13,0.13}{##1}}}
\expandafter\def\csname PY@tok@sd\endcsname{\let\PY@it=\textit\def\PY@tc##1{\textcolor[rgb]{0.73,0.13,0.13}{##1}}}
\expandafter\def\csname PY@tok@si\endcsname{\let\PY@bf=\textbf\def\PY@tc##1{\textcolor[rgb]{0.73,0.40,0.53}{##1}}}
\expandafter\def\csname PY@tok@se\endcsname{\let\PY@bf=\textbf\def\PY@tc##1{\textcolor[rgb]{0.73,0.40,0.13}{##1}}}
\expandafter\def\csname PY@tok@sr\endcsname{\def\PY@tc##1{\textcolor[rgb]{0.73,0.40,0.53}{##1}}}
\expandafter\def\csname PY@tok@ss\endcsname{\def\PY@tc##1{\textcolor[rgb]{0.10,0.09,0.49}{##1}}}
\expandafter\def\csname PY@tok@sx\endcsname{\def\PY@tc##1{\textcolor[rgb]{0.00,0.50,0.00}{##1}}}
\expandafter\def\csname PY@tok@m\endcsname{\def\PY@tc##1{\textcolor[rgb]{0.40,0.40,0.40}{##1}}}
\expandafter\def\csname PY@tok@gh\endcsname{\let\PY@bf=\textbf\def\PY@tc##1{\textcolor[rgb]{0.00,0.00,0.50}{##1}}}
\expandafter\def\csname PY@tok@gu\endcsname{\let\PY@bf=\textbf\def\PY@tc##1{\textcolor[rgb]{0.50,0.00,0.50}{##1}}}
\expandafter\def\csname PY@tok@gd\endcsname{\def\PY@tc##1{\textcolor[rgb]{0.63,0.00,0.00}{##1}}}
\expandafter\def\csname PY@tok@gi\endcsname{\def\PY@tc##1{\textcolor[rgb]{0.00,0.63,0.00}{##1}}}
\expandafter\def\csname PY@tok@gr\endcsname{\def\PY@tc##1{\textcolor[rgb]{1.00,0.00,0.00}{##1}}}
\expandafter\def\csname PY@tok@ge\endcsname{\let\PY@it=\textit}
\expandafter\def\csname PY@tok@gs\endcsname{\let\PY@bf=\textbf}
\expandafter\def\csname PY@tok@gp\endcsname{\let\PY@bf=\textbf\def\PY@tc##1{\textcolor[rgb]{0.00,0.00,0.50}{##1}}}
\expandafter\def\csname PY@tok@go\endcsname{\def\PY@tc##1{\textcolor[rgb]{0.53,0.53,0.53}{##1}}}
\expandafter\def\csname PY@tok@gt\endcsname{\def\PY@tc##1{\textcolor[rgb]{0.00,0.27,0.87}{##1}}}
\expandafter\def\csname PY@tok@err\endcsname{\def\PY@bc##1{\setlength{\fboxsep}{0pt}\fcolorbox[rgb]{1.00,0.00,0.00}{1,1,1}{\strut ##1}}}
\expandafter\def\csname PY@tok@kc\endcsname{\let\PY@bf=\textbf\def\PY@tc##1{\textcolor[rgb]{0.00,0.50,0.00}{##1}}}
\expandafter\def\csname PY@tok@kd\endcsname{\let\PY@bf=\textbf\def\PY@tc##1{\textcolor[rgb]{0.00,0.50,0.00}{##1}}}
\expandafter\def\csname PY@tok@kn\endcsname{\let\PY@bf=\textbf\def\PY@tc##1{\textcolor[rgb]{0.00,0.50,0.00}{##1}}}
\expandafter\def\csname PY@tok@kr\endcsname{\let\PY@bf=\textbf\def\PY@tc##1{\textcolor[rgb]{0.00,0.50,0.00}{##1}}}
\expandafter\def\csname PY@tok@bp\endcsname{\def\PY@tc##1{\textcolor[rgb]{0.00,0.50,0.00}{##1}}}
\expandafter\def\csname PY@tok@fm\endcsname{\def\PY@tc##1{\textcolor[rgb]{0.00,0.00,1.00}{##1}}}
\expandafter\def\csname PY@tok@vc\endcsname{\def\PY@tc##1{\textcolor[rgb]{0.10,0.09,0.49}{##1}}}
\expandafter\def\csname PY@tok@vg\endcsname{\def\PY@tc##1{\textcolor[rgb]{0.10,0.09,0.49}{##1}}}
\expandafter\def\csname PY@tok@vi\endcsname{\def\PY@tc##1{\textcolor[rgb]{0.10,0.09,0.49}{##1}}}
\expandafter\def\csname PY@tok@vm\endcsname{\def\PY@tc##1{\textcolor[rgb]{0.10,0.09,0.49}{##1}}}
\expandafter\def\csname PY@tok@sa\endcsname{\def\PY@tc##1{\textcolor[rgb]{0.73,0.13,0.13}{##1}}}
\expandafter\def\csname PY@tok@sb\endcsname{\def\PY@tc##1{\textcolor[rgb]{0.73,0.13,0.13}{##1}}}
\expandafter\def\csname PY@tok@sc\endcsname{\def\PY@tc##1{\textcolor[rgb]{0.73,0.13,0.13}{##1}}}
\expandafter\def\csname PY@tok@dl\endcsname{\def\PY@tc##1{\textcolor[rgb]{0.73,0.13,0.13}{##1}}}
\expandafter\def\csname PY@tok@s2\endcsname{\def\PY@tc##1{\textcolor[rgb]{0.73,0.13,0.13}{##1}}}
\expandafter\def\csname PY@tok@sh\endcsname{\def\PY@tc##1{\textcolor[rgb]{0.73,0.13,0.13}{##1}}}
\expandafter\def\csname PY@tok@s1\endcsname{\def\PY@tc##1{\textcolor[rgb]{0.73,0.13,0.13}{##1}}}
\expandafter\def\csname PY@tok@mb\endcsname{\def\PY@tc##1{\textcolor[rgb]{0.40,0.40,0.40}{##1}}}
\expandafter\def\csname PY@tok@mf\endcsname{\def\PY@tc##1{\textcolor[rgb]{0.40,0.40,0.40}{##1}}}
\expandafter\def\csname PY@tok@mh\endcsname{\def\PY@tc##1{\textcolor[rgb]{0.40,0.40,0.40}{##1}}}
\expandafter\def\csname PY@tok@mi\endcsname{\def\PY@tc##1{\textcolor[rgb]{0.40,0.40,0.40}{##1}}}
\expandafter\def\csname PY@tok@il\endcsname{\def\PY@tc##1{\textcolor[rgb]{0.40,0.40,0.40}{##1}}}
\expandafter\def\csname PY@tok@mo\endcsname{\def\PY@tc##1{\textcolor[rgb]{0.40,0.40,0.40}{##1}}}
\expandafter\def\csname PY@tok@ch\endcsname{\let\PY@it=\textit\def\PY@tc##1{\textcolor[rgb]{0.25,0.50,0.50}{##1}}}
\expandafter\def\csname PY@tok@cm\endcsname{\let\PY@it=\textit\def\PY@tc##1{\textcolor[rgb]{0.25,0.50,0.50}{##1}}}
\expandafter\def\csname PY@tok@cpf\endcsname{\let\PY@it=\textit\def\PY@tc##1{\textcolor[rgb]{0.25,0.50,0.50}{##1}}}
\expandafter\def\csname PY@tok@c1\endcsname{\let\PY@it=\textit\def\PY@tc##1{\textcolor[rgb]{0.25,0.50,0.50}{##1}}}
\expandafter\def\csname PY@tok@cs\endcsname{\let\PY@it=\textit\def\PY@tc##1{\textcolor[rgb]{0.25,0.50,0.50}{##1}}}

\def\PYZbs{\char`\\}
\def\PYZus{\char`\_}
\def\PYZob{\char`\{}
\def\PYZcb{\char`\}}
\def\PYZca{\char`\^}
\def\PYZam{\char`\&}
\def\PYZlt{\char`\<}
\def\PYZgt{\char`\>}
\def\PYZsh{\char`\#}
\def\PYZpc{\char`\%}
\def\PYZdl{\char`\$}
\def\PYZhy{\char`\-}
\def\PYZsq{\char`\'}
\def\PYZdq{\char`\"}
\def\PYZti{\char`\~}
% for compatibility with earlier versions
\def\PYZat{@}
\def\PYZlb{[}
\def\PYZrb{]}
\makeatother


    % Exact colors from NB
    \definecolor{incolor}{rgb}{0.0, 0.0, 0.5}
    \definecolor{outcolor}{rgb}{0.545, 0.0, 0.0}



    
    % Prevent overflowing lines due to hard-to-break entities
    \sloppy 
    % Setup hyperref package
    \hypersetup{
      breaklinks=true,  % so long urls are correctly broken across lines
      colorlinks=true,
      urlcolor=urlcolor,
      linkcolor=linkcolor,
      citecolor=citecolor,
      }
    % Slightly bigger margins than the latex defaults
    
    \geometry{verbose,tmargin=1in,bmargin=1in,lmargin=1in,rmargin=1in}
    
    

    \begin{document}
    
    
    \maketitle
    
    

    
    Lab: Working with a real world data-set using SQL and Python

    \section{Introduction}\label{introduction}

This notebook shows how to work with a real world dataset using SQL and
Python. In this lab you will: 1. Understand the dataset for Chicago
Public School level performance 1. Store the dataset in an Db2 database
on IBM Cloud instance 1. Retrieve metadata about tables and columns and
query data from mixed case columns 1. Solve example problems to practice
your SQL skills including using built-in database functions

    \subsection{Chicago Public Schools - Progress Report Cards
(2011-2012)}\label{chicago-public-schools---progress-report-cards-2011-2012}

The city of Chicago released a dataset showing all school level
performance data used to create School Report Cards for the 2011-2012
school year. The dataset is available from the Chicago Data Portal:
https://data.cityofchicago.org/Education/Chicago-Public-Schools-Progress-Report-Cards-2011-/9xs2-f89t

This dataset includes a large number of metrics. Start by familiarizing
yourself with the types of metrics in the database:
https://data.cityofchicago.org/api/assets/AAD41A13-BE8A-4E67-B1F5-86E711E09D5F?download=true

Now download a static copy of this database and review some of its
contents:
https://ibm.box.com/shared/static/0g7kbanvn5l2gt2qu38ukooatnjqyuys.csv

    \subsubsection{Store the dataset in a
Table}\label{store-the-dataset-in-a-table}

In many cases the dataset to be analyzed is available as a .CSV (comma
separated values) file, perhaps on the internet. To analyze the data
using SQL, it first needs to be stored in the database.

While it is easier to read the dataset into a Pandas dataframe and then
PERSIST it into the database as we saw in the previous lab, it results
in mapping to default datatypes which may not be optimal for SQL
querying. For example a long textual field may map to a CLOB instead of
a VARCHAR.

Therefore, \textbf{it is highly recommended to manually load the table
using the database console LOAD tool, as indicated in Week 2 Lab 1 Part
II}. The only difference with that lab is that in Step 5 of the
instructions you will need to click on create "(+) New Table" and
specify the name of the table you want to create and then click "Next".

\subparagraph{\texorpdfstring{Now open the Db2 console, open the LOAD
tool, Select / Drag the .CSV file for the CHICAGO PUBLIC SCHOOLS dataset
and load the dataset into a new table called
\textbf{SCHOOLS}.}{Now open the Db2 console, open the LOAD tool, Select / Drag the .CSV file for the CHICAGO PUBLIC SCHOOLS dataset and load the dataset into a new table called SCHOOLS.}}\label{now-open-the-db2-console-open-the-load-tool-select-drag-the-.csv-file-for-the-chicago-public-schools-dataset-and-load-the-dataset-into-a-new-table-called-schools.}

    \subsubsection{Connect to the database}\label{connect-to-the-database}

Let us now load the ipython-sql extension and establish a connection
with the database

    \begin{Verbatim}[commandchars=\\\{\}]
{\color{incolor}In [{\color{incolor}1}]:} \PY{o}{\PYZpc{}}\PY{k}{load\PYZus{}ext} sql
\end{Verbatim}


    \begin{Verbatim}[commandchars=\\\{\}]
{\color{incolor}In [{\color{incolor}2}]:} \PY{c+c1}{\PYZsh{} Enter the connection string for your Db2 on Cloud database instance below}
        \PY{c+c1}{\PYZsh{} \PYZpc{}sql ibm\PYZus{}db\PYZus{}sa://my\PYZhy{}username:my\PYZhy{}password@my\PYZhy{}hostname:my\PYZhy{}port/my\PYZhy{}db\PYZhy{}name/}
        \PY{o}{\PYZpc{}}\PY{k}{sql} ibm\PYZus{}db\PYZus{}sa://hcg90289:bw10zlmgx7crb\PYZhy{}gb@dashdb\PYZhy{}txn\PYZhy{}sbox\PYZhy{}yp\PYZhy{}dal09\PYZhy{}04.services.dal.bluemix.net:50000/BLUDB
\end{Verbatim}


\begin{Verbatim}[commandchars=\\\{\}]
{\color{outcolor}Out[{\color{outcolor}2}]:} 'Connected: hcg90289@BLUDB'
\end{Verbatim}
            
    \subsubsection{Query the database system catalog to retrieve table
metadata}\label{query-the-database-system-catalog-to-retrieve-table-metadata}

\subparagraph{You can verify that the table creation was successful by
retrieving the list of all tables in your schema and checking whether
the SCHOOLS table was
created}\label{you-can-verify-that-the-table-creation-was-successful-by-retrieving-the-list-of-all-tables-in-your-schema-and-checking-whether-the-schools-table-was-created}

    \begin{Verbatim}[commandchars=\\\{\}]
{\color{incolor}In [{\color{incolor}12}]:} \PY{c+c1}{\PYZsh{} type in your query to retrieve list of all tables in the database for your db2 schema (username)}
         \PY{o}{\PYZpc{}}\PY{k}{sql} select TABSCHEMA, TABNAME, CREATE\PYZus{}TIME from SYSCAT.TABLES where TABSCHEMA=\PYZsq{}Db2\PYZhy{}bo\PYZsq{}
         \PY{o}{\PYZpc{}}\PY{k}{sql} select * from SYSCAT.TABLES where TABNAME = \PYZsq{}SCHOOLS\PYZsq{}
         \PY{c+c1}{\PYZsh{} \PYZpc{}sql select TABSCHEMA, TABNAME, CREATE\PYZus{}TIME from SYSCAT.TABLES \PYZbs{}}
         \PY{c+c1}{\PYZsh{}      where TABSCHEMA not in (\PYZsq{}SYSIBM\PYZsq{}, \PYZsq{}SYSCAT\PYZsq{}, \PYZsq{}SYSSTAT\PYZsq{}, \PYZsq{}SYSIBMADM\PYZsq{}, \PYZsq{}SYSTOOLS\PYZsq{}, \PYZsq{}SYSPUBLIC\PYZsq{})}
\end{Verbatim}


    \begin{Verbatim}[commandchars=\\\{\}]
 * ibm\_db\_sa://hcg90289:***@dashdb-txn-sbox-yp-dal09-04.services.dal.bluemix.net:50000/BLUDB
Done.
 * ibm\_db\_sa://hcg90289:***@dashdb-txn-sbox-yp-dal09-04.services.dal.bluemix.net:50000/BLUDB
Done.

    \end{Verbatim}

\begin{Verbatim}[commandchars=\\\{\}]
{\color{outcolor}Out[{\color{outcolor}12}]:} [('HCG90289', 'SCHOOLS', 'HCG90289', 'U', 'T', 'N', None, None, None, None, datetime.datetime(2018, 11, 28, 23, 57, 31, 337007), datetime.datetime(2018, 11, 28, 23, 57, 31, 337007), datetime.datetime(2018, 11, 28, 23, 57, 31, 337007), datetime.datetime(2018, 11, 29, 0, 2, 18, 951928), 79, 4, 788, 566, 15, 0, 16, -1, -1, -1, 0, 'hcg90289space1', None, None, 0, 0, 0, 0, 0, 0, 0, 'N', 'YYYYYYYYYYYYYYYYYYYYYYYYYYYYYYYY', 1, ' ', '0', -1, 'N', ' ', None, 'R', ' ', 'N', '                                ', None, 'N', ' ', 'F', None, 0, 'N', 999, 0, 0.0, 831, 0.0, None, 1208, 'SYSIBM', 'IDENTITY', 'SYSIBM', 'IDENTITY', ' ', 0, datetime.datetime(2018, 11, 28, 23, 57, 31, 337007), 0, ' ', None, None, 'N', 'HCG90289', ' ', ' ', ' ', datetime.date(2018, 12, 1), ' ', 'N', 'R', 'N', -1.0, None)]
\end{Verbatim}
            
    Double-click \textbf{here} for a hint

    Double-click \textbf{here} for the solution.

    \subsubsection{Query the database system catalog to retrieve column
metadata}\label{query-the-database-system-catalog-to-retrieve-column-metadata}

\subparagraph{The SCHOOLS table contains a large number of columns. How
many columns does this table
have?}\label{the-schools-table-contains-a-large-number-of-columns.-how-many-columns-does-this-table-have}

    \begin{Verbatim}[commandchars=\\\{\}]
{\color{incolor}In [{\color{incolor}6}]:} \PY{c+c1}{\PYZsh{} type in your query to retrieve the number of columns in the SCHOOLS table}
        \PY{o}{\PYZpc{}}\PY{k}{sql} select count(*) from SYSCAT.COLUMNS where TABNAME = \PYZsq{}SCHOOLS\PYZsq{}
\end{Verbatim}


    \begin{Verbatim}[commandchars=\\\{\}]
 * ibm\_db\_sa://hcg90289:***@dashdb-txn-sbox-yp-dal09-04.services.dal.bluemix.net:50000/BLUDB
Done.

    \end{Verbatim}

\begin{Verbatim}[commandchars=\\\{\}]
{\color{outcolor}Out[{\color{outcolor}6}]:} [(Decimal('79'),)]
\end{Verbatim}
            
    Double-click \textbf{here} for a hint

    Double-click \textbf{here} for the solution.

    Now retrieve the the list of columns in SCHOOLS table and their column
type (datatype) and length.

    \begin{Verbatim}[commandchars=\\\{\}]
{\color{incolor}In [{\color{incolor}9}]:} \PY{c+c1}{\PYZsh{} type in your query to retrieve all column names in the SCHOOLS table along with their datatypes and length}
        \PY{c+c1}{\PYZsh{} \PYZpc{}sql select COLNAME, TYPENAME, LENGTH from SYSCAT.COLUMNS where TABNAME = \PYZsq{}SCHOOLS\PYZsq{}}
        \PY{o}{\PYZpc{}}\PY{k}{sql} select distinct(NAME), COLTYPE, LENGTH from SYSIBM.SYSCOLUMNS where TBNAME = \PYZsq{}SCHOOLS\PYZsq{}
\end{Verbatim}


    \begin{Verbatim}[commandchars=\\\{\}]
 * ibm\_db\_sa://hcg90289:***@dashdb-txn-sbox-yp-dal09-04.services.dal.bluemix.net:50000/BLUDB
Done.

    \end{Verbatim}

\begin{Verbatim}[commandchars=\\\{\}]
{\color{outcolor}Out[{\color{outcolor}9}]:} [('10th Grade PLAN (2009)', 'VARCHAR ', 4),
         ('10th Grade PLAN (2010)', 'VARCHAR ', 4),
         ('11th Grade Average ACT (2011)', 'VARCHAR ', 4),
         ('9th Grade EXPLORE (2009)', 'VARCHAR ', 4),
         ('9th Grade EXPLORE (2010)', 'VARCHAR ', 4),
         ('Adequate\_Yearly\_Progress\_Made\_', 'VARCHAR ', 3),
         ('Average\_Student\_Attendance', 'DECIMAL ', 4),
         ('Average\_Teacher\_Attendance', 'DECIMAL ', 4),
         ('CPS\_Performance\_Policy\_Level', 'VARCHAR ', 15),
         ('CPS\_Performance\_Policy\_Status', 'VARCHAR ', 16),
         ('City', 'VARCHAR ', 7),
         ('Collaborative\_Name', 'VARCHAR ', 34),
         ('College\_Eligibility\_\_', 'VARCHAR ', 4),
         ('College\_Enrollment\_Rate\_\_', 'VARCHAR ', 4),
         ('College\_Enrollment\_\_number\_of\_students\_', 'SMALLINT', 2),
         ('Community\_Area\_Name', 'VARCHAR ', 22),
         ('Community\_Area\_Number', 'SMALLINT', 2),
         ('Elementary, Middle, or High School', 'VARCHAR ', 2),
         ('Environment\_Icon', 'VARCHAR ', 11),
         ('Environment\_Score', 'SMALLINT', 2),
         ('Family\_Involvement\_Icon', 'VARCHAR ', 11),
         ('Family\_Involvement\_Score', 'VARCHAR ', 3),
         ('Freshman\_on\_Track\_Rate\_\_', 'VARCHAR ', 4),
         ('General\_Services\_Route', 'SMALLINT', 2),
         ('Gr3\_5\_Grade\_Level\_Math\_\_', 'VARCHAR ', 4),
         ('Gr3\_5\_Grade\_Level\_Read\_\_', 'VARCHAR ', 4),
         ('Gr3\_5\_Keep\_Pace\_Math\_\_', 'VARCHAR ', 4),
         ('Gr3\_5\_Keep\_Pace\_Read\_\_', 'VARCHAR ', 4),
         ('Gr6\_8\_Grade\_Level\_Math\_\_', 'VARCHAR ', 4),
         ('Gr6\_8\_Grade\_Level\_Read\_\_', 'VARCHAR ', 4),
         ('Gr6\_8\_Keep\_Pace\_Math\_', 'VARCHAR ', 4),
         ('Gr6\_8\_Keep\_Pace\_Read\_\_', 'VARCHAR ', 4),
         ('Gr\_8\_Explore\_Math\_\_', 'VARCHAR ', 4),
         ('Gr\_8\_Explore\_Read\_\_', 'VARCHAR ', 4),
         ('Graduation\_Rate\_\_', 'VARCHAR ', 4),
         ('Healthy\_Schools\_Certified\_', 'VARCHAR ', 3),
         ('ISAT\_Exceeding\_Math\_\_', 'DECIMAL ', 4),
         ('ISAT\_Exceeding\_Reading\_\_', 'DECIMAL ', 4),
         ('ISAT\_Value\_Add\_Color\_Math', 'VARCHAR ', 6),
         ('ISAT\_Value\_Add\_Color\_Read', 'VARCHAR ', 6),
         ('ISAT\_Value\_Add\_Math', 'DECIMAL ', 3),
         ('ISAT\_Value\_Add\_Read', 'DECIMAL ', 3),
         ('Individualized\_Education\_Program\_Compliance\_Rate', 'DECIMAL ', 5),
         ('Instruction\_Icon', 'VARCHAR ', 11),
         ('Instruction\_Score', 'SMALLINT', 2),
         ('Latitude', 'DECIMAL ', 18),
         ('Leaders\_Icon', 'VARCHAR ', 11),
         ('Leaders\_Score', 'VARCHAR ', 3),
         ('Link', 'VARCHAR ', 78),
         ('Location', 'VARCHAR ', 27),
         ('Longitude', 'DECIMAL ', 18),
         ('Name\_of\_School', 'VARCHAR ', 65),
         ('Net\_Change\_EXPLORE\_and\_PLAN', 'VARCHAR ', 3),
         ('Net\_Change\_PLAN\_and\_ACT', 'VARCHAR ', 3),
         ('Network\_Manager', 'VARCHAR ', 40),
         ('Parent\_Engagement\_Icon', 'VARCHAR ', 7),
         ('Parent\_Engagement\_Score', 'VARCHAR ', 3),
         ('Parent\_Environment\_Icon', 'VARCHAR ', 7),
         ('Parent\_Environment\_Score', 'VARCHAR ', 3),
         ('Phone\_Number', 'VARCHAR ', 14),
         ('Pk\_2\_Literacy\_\_', 'VARCHAR ', 4),
         ('Pk\_2\_Math\_\_', 'VARCHAR ', 4),
         ('Police\_District', 'SMALLINT', 2),
         ('RCDTS\_Code', 'BIGINT  ', 8),
         ('Rate\_of\_Misconducts\_\_per\_100\_students\_', 'DECIMAL ', 5),
         ('Safety\_Icon', 'VARCHAR ', 11),
         ('Safety\_Score', 'SMALLINT', 2),
         ('School\_ID', 'INTEGER ', 4),
         ('State', 'VARCHAR ', 2),
         ('Street\_Address', 'VARCHAR ', 30),
         ('Students\_Passing\_\_Algebra\_\_', 'VARCHAR ', 4),
         ('Students\_Taking\_\_Algebra\_\_', 'VARCHAR ', 4),
         ('Teachers\_Icon', 'VARCHAR ', 11),
         ('Teachers\_Score', 'VARCHAR ', 3),
         ('Track\_Schedule', 'VARCHAR ', 12),
         ('Ward', 'SMALLINT', 2),
         ('X\_COORDINATE', 'DECIMAL ', 13),
         ('Y\_COORDINATE', 'DECIMAL ', 13),
         ('ZIP\_Code', 'INTEGER ', 4)]
\end{Verbatim}
            
    Double-click \textbf{here} for the solution.

    \subsubsection{Questions}\label{questions}

\begin{enumerate}
\def\labelenumi{\arabic{enumi}.}
\tightlist
\item
  Is the column name for the "SCHOOL ID" attribute in upper or mixed
  case?
\item
  What is the "Community Area Name" field called in your table? Have the
  spaces " " between the words been replaced by some other character?
\item
  Have the paranthesis (round brackets) in the "College Enrollment
  (number of students)" attribute been replaced by some other character?
\item
  Are there any columns in whose names the spaces and paranthesis (round
  brackets) have not been replaced by the underscore character "\_"?
\end{enumerate}

    \subsection{Problems}\label{problems}

\subsubsection{Problem 1}\label{problem-1}

\subparagraph{How many Elementary Schools are in the
dataset?}\label{how-many-elementary-schools-are-in-the-dataset}

    \begin{Verbatim}[commandchars=\\\{\}]
{\color{incolor}In [{\color{incolor}13}]:} \PY{o}{\PYZpc{}}\PY{k}{sql} select count(*) from SCHOOLS where \PYZdq{}Elementary, Middle, or High School\PYZdq{} = \PYZsq{}ES\PYZsq{}
\end{Verbatim}


    \begin{Verbatim}[commandchars=\\\{\}]
 * ibm\_db\_sa://hcg90289:***@dashdb-txn-sbox-yp-dal09-04.services.dal.bluemix.net:50000/BLUDB
Done.

    \end{Verbatim}

\begin{Verbatim}[commandchars=\\\{\}]
{\color{outcolor}Out[{\color{outcolor}13}]:} [(Decimal('462'),)]
\end{Verbatim}
            
    Double-click \textbf{here} for a hint

    Double-click \textbf{here} for another hint

    Double-click \textbf{here} for the solution.

    \subsubsection{Problem 2}\label{problem-2}

\subparagraph{What is the highest Safety
Score?}\label{what-is-the-highest-safety-score}

    \begin{Verbatim}[commandchars=\\\{\}]
{\color{incolor}In [{\color{incolor}14}]:} \PY{o}{\PYZpc{}}\PY{k}{sql} select MAX(\PYZdq{}Safety\PYZus{}Score\PYZdq{}) AS MAX\PYZus{}SAFETY\PYZus{}SCORE from SCHOOLS
\end{Verbatim}


    \begin{Verbatim}[commandchars=\\\{\}]
 * ibm\_db\_sa://hcg90289:***@dashdb-txn-sbox-yp-dal09-04.services.dal.bluemix.net:50000/BLUDB
Done.

    \end{Verbatim}

\begin{Verbatim}[commandchars=\\\{\}]
{\color{outcolor}Out[{\color{outcolor}14}]:} [(99,)]
\end{Verbatim}
            
    Double-click \textbf{here} for a hint

    Double-click \textbf{here} for the solution.

    \subsubsection{Problem 3}\label{problem-3}

\subparagraph{Which schools have highest Safety
Score?}\label{which-schools-have-highest-safety-score}

    \begin{Verbatim}[commandchars=\\\{\}]
{\color{incolor}In [{\color{incolor}15}]:} \PY{o}{\PYZpc{}}\PY{k}{sql} select \PYZdq{}Name\PYZus{}of\PYZus{}School\PYZdq{}, \PYZdq{}Safety\PYZus{}Score\PYZdq{} from SCHOOLS where \PYZbs{}
           \PY{l+s+s2}{\PYZdq{}}\PY{l+s+s2}{Safety\PYZus{}Score}\PY{l+s+s2}{\PYZdq{}}\PY{o}{=} \PY{p}{(}\PY{n}{select} \PY{n}{MAX}\PY{p}{(}\PY{l+s+s2}{\PYZdq{}}\PY{l+s+s2}{Safety\PYZus{}Score}\PY{l+s+s2}{\PYZdq{}}\PY{p}{)} \PY{k+kn}{from} \PY{n+nn}{SCHOOLS}\PY{p}{)}
\end{Verbatim}


    \begin{Verbatim}[commandchars=\\\{\}]
 * ibm\_db\_sa://hcg90289:***@dashdb-txn-sbox-yp-dal09-04.services.dal.bluemix.net:50000/BLUDB
Done.

    \end{Verbatim}

\begin{Verbatim}[commandchars=\\\{\}]
{\color{outcolor}Out[{\color{outcolor}15}]:} [('Abraham Lincoln Elementary School', 99),
          ('Alexander Graham Bell Elementary School', 99),
          ('Annie Keller Elementary Gifted Magnet School', 99),
          ('Augustus H Burley Elementary School', 99),
          ('Edgar Allan Poe Elementary Classical School', 99),
          ('Edgebrook Elementary School', 99),
          ('Ellen Mitchell Elementary School', 99),
          ('James E McDade Elementary Classical School', 99),
          ('James G Blaine Elementary School', 99),
          ('LaSalle Elementary Language Academy', 99),
          ('Mary E Courtenay Elementary Language Arts Center', 99),
          ('Northside College Preparatory High School', 99),
          ('Northside Learning Center High School', 99),
          ('Norwood Park Elementary School', 99),
          ('Oriole Park Elementary School', 99),
          ('Sauganash Elementary School', 99),
          ('Stephen Decatur Classical Elementary School', 99),
          ('Talman Elementary School', 99),
          ('Wildwood Elementary School', 99)]
\end{Verbatim}
            
    Double-click \textbf{here} for the solution.

    \subsubsection{Problem 4}\label{problem-4}

\subparagraph{What are the top 10 schools with the highest "Average
Student
Attendance"?}\label{what-are-the-top-10-schools-with-the-highest-average-student-attendance}

    \begin{Verbatim}[commandchars=\\\{\}]
{\color{incolor}In [{\color{incolor}16}]:} \PY{o}{\PYZpc{}}\PY{k}{sql} select \PYZdq{}Name\PYZus{}of\PYZus{}School\PYZdq{}, \PYZdq{}Average\PYZus{}Student\PYZus{}Attendance\PYZdq{} from SCHOOLS \PYZbs{}
             \PY{n}{order} \PY{n}{by} \PY{l+s+s2}{\PYZdq{}}\PY{l+s+s2}{Average\PYZus{}Student\PYZus{}Attendance}\PY{l+s+s2}{\PYZdq{}} \PY{n}{desc} \PY{n}{nulls} \PY{n}{last} \PY{n}{limit} \PY{l+m+mi}{10} 
\end{Verbatim}


    \begin{Verbatim}[commandchars=\\\{\}]
 * ibm\_db\_sa://hcg90289:***@dashdb-txn-sbox-yp-dal09-04.services.dal.bluemix.net:50000/BLUDB
Done.

    \end{Verbatim}

\begin{Verbatim}[commandchars=\\\{\}]
{\color{outcolor}Out[{\color{outcolor}16}]:} [('John Charles Haines Elementary School', Decimal('98.4')),
          ('James Ward Elementary School', Decimal('97.8')),
          ('Edgar Allan Poe Elementary Classical School', Decimal('97.6')),
          ('Orozco Fine Arts \& Sciences Elementary School', Decimal('97.6')),
          ('Rachel Carson Elementary School', Decimal('97.6')),
          ('Annie Keller Elementary Gifted Magnet School', Decimal('97.5')),
          ('Andrew Jackson Elementary Language Academy', Decimal('97.4')),
          ('Lenart Elementary Regional Gifted Center', Decimal('97.4')),
          ('Disney II Magnet School', Decimal('97.3')),
          ('John H Vanderpoel Elementary Magnet School', Decimal('97.2'))]
\end{Verbatim}
            
    Double-click \textbf{here} for the solution.

    \subsubsection{Problem 5}\label{problem-5}

\subparagraph{Retrieve the list of 5 Schools with the lowest Average
Student Attendance sorted in ascending order based on
attendance}\label{retrieve-the-list-of-5-schools-with-the-lowest-average-student-attendance-sorted-in-ascending-order-based-on-attendance}

    \begin{Verbatim}[commandchars=\\\{\}]
{\color{incolor}In [{\color{incolor}17}]:} \PY{o}{\PYZpc{}}\PY{k}{sql} SELECT \PYZdq{}Name\PYZus{}of\PYZus{}School\PYZdq{}, \PYZdq{}Average\PYZus{}Student\PYZus{}Attendance\PYZdq{}  \PYZbs{}
              \PY{k+kn}{from} \PY{n+nn}{SCHOOLS} \PYZbs{}
              \PY{n}{order} \PY{n}{by} \PY{l+s+s2}{\PYZdq{}}\PY{l+s+s2}{Average\PYZus{}Student\PYZus{}Attendance}\PY{l+s+s2}{\PYZdq{}} \PYZbs{}
              \PY{n}{fetch} \PY{n}{first} \PY{l+m+mi}{5} \PY{n}{rows} \PY{n}{only}
\end{Verbatim}


    \begin{Verbatim}[commandchars=\\\{\}]
 * ibm\_db\_sa://hcg90289:***@dashdb-txn-sbox-yp-dal09-04.services.dal.bluemix.net:50000/BLUDB
Done.

    \end{Verbatim}

\begin{Verbatim}[commandchars=\\\{\}]
{\color{outcolor}Out[{\color{outcolor}17}]:} [('Richard T Crane Technical Preparatory High School', Decimal('57.9')),
          ('Barbara Vick Early Childhood \& Family Center', Decimal('60.9')),
          ('Dyett High School', Decimal('62.5')),
          ('Wendell Phillips Academy High School', Decimal('63.0')),
          ('Orr Academy High School', Decimal('66.3'))]
\end{Verbatim}
            
    Double-click \textbf{here} for the solution.

    \subsubsection{Problem 6}\label{problem-6}

\subparagraph{Now remove the '\%' sign from the above result set for
Average Student Attendance
column}\label{now-remove-the-sign-from-the-above-result-set-for-average-student-attendance-column}

    \begin{Verbatim}[commandchars=\\\{\}]
{\color{incolor}In [{\color{incolor}18}]:} \PY{o}{\PYZpc{}}\PY{k}{sql} SELECT \PYZdq{}Name\PYZus{}of\PYZus{}School\PYZdq{}, REPLACE(\PYZdq{}Average\PYZus{}Student\PYZus{}Attendance\PYZdq{}, \PYZsq{}\PYZpc{}\PYZsq{}, \PYZsq{}\PYZsq{}) \PYZbs{}
              \PY{k+kn}{from} \PY{n+nn}{SCHOOLS} \PYZbs{}
              \PY{n}{order} \PY{n}{by} \PY{l+s+s2}{\PYZdq{}}\PY{l+s+s2}{Average\PYZus{}Student\PYZus{}Attendance}\PY{l+s+s2}{\PYZdq{}} \PYZbs{}
              \PY{n}{fetch} \PY{n}{first} \PY{l+m+mi}{5} \PY{n}{rows} \PY{n}{only}
\end{Verbatim}


    \begin{Verbatim}[commandchars=\\\{\}]
 * ibm\_db\_sa://hcg90289:***@dashdb-txn-sbox-yp-dal09-04.services.dal.bluemix.net:50000/BLUDB
Done.

    \end{Verbatim}

\begin{Verbatim}[commandchars=\\\{\}]
{\color{outcolor}Out[{\color{outcolor}18}]:} [('Richard T Crane Technical Preparatory High School', '57.9'),
          ('Barbara Vick Early Childhood \& Family Center', '60.9'),
          ('Dyett High School', '62.5'),
          ('Wendell Phillips Academy High School', '63.0'),
          ('Orr Academy High School', '66.3')]
\end{Verbatim}
            
    Double-click \textbf{here} for a hint

    Double-click \textbf{here} for the solution.

    \subsubsection{Problem 7}\label{problem-7}

\subparagraph{Which Schools have Average Student Attendance lower than
70\%?}\label{which-schools-have-average-student-attendance-lower-than-70}

    \begin{Verbatim}[commandchars=\\\{\}]
{\color{incolor}In [{\color{incolor}19}]:} \PY{o}{\PYZpc{}}\PY{k}{sql} SELECT \PYZdq{}Name\PYZus{}of\PYZus{}School\PYZdq{}, \PYZdq{}Average\PYZus{}Student\PYZus{}Attendance\PYZdq{}  \PYZbs{}
              \PY{k+kn}{from} \PY{n+nn}{SCHOOLS} \PYZbs{}
              \PY{n}{where} \PY{n}{CAST} \PY{p}{(} \PY{n}{REPLACE}\PY{p}{(}\PY{l+s+s2}{\PYZdq{}}\PY{l+s+s2}{Average\PYZus{}Student\PYZus{}Attendance}\PY{l+s+s2}{\PYZdq{}}\PY{p}{,} \PY{l+s+s1}{\PYZsq{}}\PY{l+s+s1}{\PYZpc{}}\PY{l+s+s1}{\PYZsq{}}\PY{p}{,} \PY{l+s+s1}{\PYZsq{}}\PY{l+s+s1}{\PYZsq{}}\PY{p}{)} \PY{n}{AS} \PY{n}{DOUBLE} \PY{p}{)} \PY{o}{\PYZlt{}} \PY{l+m+mi}{70} \PYZbs{}
              \PY{n}{order} \PY{n}{by} \PY{l+s+s2}{\PYZdq{}}\PY{l+s+s2}{Average\PYZus{}Student\PYZus{}Attendance}\PY{l+s+s2}{\PYZdq{}}
\end{Verbatim}


    \begin{Verbatim}[commandchars=\\\{\}]
 * ibm\_db\_sa://hcg90289:***@dashdb-txn-sbox-yp-dal09-04.services.dal.bluemix.net:50000/BLUDB
Done.

    \end{Verbatim}

\begin{Verbatim}[commandchars=\\\{\}]
{\color{outcolor}Out[{\color{outcolor}19}]:} [('Richard T Crane Technical Preparatory High School', Decimal('57.9')),
          ('Barbara Vick Early Childhood \& Family Center', Decimal('60.9')),
          ('Dyett High School', Decimal('62.5')),
          ('Wendell Phillips Academy High School', Decimal('63.0')),
          ('Orr Academy High School', Decimal('66.3')),
          ('Manley Career Academy High School', Decimal('66.8')),
          ('Chicago Vocational Career Academy High School', Decimal('68.8')),
          ('Roberto Clemente Community Academy High School', Decimal('69.6'))]
\end{Verbatim}
            
    Double-click \textbf{here} for a hint

    Double-click \textbf{here} for another hint

    Double-click \textbf{here} for the solution.

    \subsubsection{Problem 8}\label{problem-8}

\subparagraph{\texorpdfstring{Get the total \textbf{College Enrollment
(number of students)} for each Community
Area}{Get the total College Enrollment (number of students) for each Community Area}}\label{get-the-total-college-enrollment-number-of-students-for-each-community-area}

    \begin{Verbatim}[commandchars=\\\{\}]
{\color{incolor}In [{\color{incolor}20}]:} \PY{o}{\PYZpc{}}\PY{k}{sql} select \PYZdq{}Community\PYZus{}Area\PYZus{}Name\PYZdq{}, sum(\PYZdq{}College\PYZus{}Enrollment\PYZus{}\PYZus{}number\PYZus{}of\PYZus{}students\PYZus{}\PYZdq{}) AS TOTAL\PYZus{}ENROLLMENT \PYZbs{}
            \PY{k+kn}{from} \PY{n+nn}{SCHOOLS} \PYZbs{}
            \PY{n}{group} \PY{n}{by} \PY{l+s+s2}{\PYZdq{}}\PY{l+s+s2}{Community\PYZus{}Area\PYZus{}Name}\PY{l+s+s2}{\PYZdq{}} 
\end{Verbatim}


    \begin{Verbatim}[commandchars=\\\{\}]
 * ibm\_db\_sa://hcg90289:***@dashdb-txn-sbox-yp-dal09-04.services.dal.bluemix.net:50000/BLUDB
Done.

    \end{Verbatim}

\begin{Verbatim}[commandchars=\\\{\}]
{\color{outcolor}Out[{\color{outcolor}20}]:} [('ALBANY PARK', 6864),
          ('ARCHER HEIGHTS', 4823),
          ('ARMOUR SQUARE', 1458),
          ('ASHBURN', 6483),
          ('AUBURN GRESHAM', 4175),
          ('AUSTIN', 10933),
          ('AVALON PARK', 1522),
          ('AVONDALE', 3640),
          ('BELMONT CRAGIN', 14386),
          ('BEVERLY', 1636),
          ('BRIDGEPORT', 3167),
          ('BRIGHTON PARK', 9647),
          ('BURNSIDE', 549),
          ('CALUMET HEIGHTS', 1568),
          ('CHATHAM', 5042),
          ('CHICAGO LAWN', 7086),
          ('CLEARING', 2085),
          ('DOUGLAS', 4670),
          ('DUNNING', 4568),
          ('EAST GARFIELD PARK', 5337),
          ('EAST SIDE', 5305),
          ('EDGEWATER', 4600),
          ('EDISON PARK', 910),
          ('ENGLEWOOD', 6832),
          ('FOREST GLEN', 1431),
          ('FULLER PARK', 531),
          ('GAGE PARK', 9915),
          ('GARFIELD RIDGE', 4552),
          ('GRAND BOULEVARD', 2809),
          ('GREATER GRAND CROSSING', 4051),
          ('HEGEWISCH', 963),
          ('HERMOSA', 3975),
          ('HUMBOLDT PARK', 8620),
          ('HYDE PARK', 1930),
          ('IRVING PARK', 7764),
          ('JEFFERSON PARK', 1755),
          ('KENWOOD', 4287),
          ('LAKE VIEW', 7055),
          ('LINCOLN PARK', 5615),
          ('LINCOLN SQUARE', 4132),
          ('LOGAN SQUARE', 7351),
          ('LOOP', 871),
          ('LOWER WEST SIDE', 7257),
          ('MCKINLEY PARK', 1552),
          ('MONTCLARE', 1317),
          ('MORGAN PARK', 3271),
          ('MOUNT GREENWOOD', 2091),
          ('NEAR NORTH SIDE', 3362),
          ('NEAR SOUTH SIDE', 1378),
          ('NEAR WEST SIDE', 7975),
          ('NEW CITY', 7922),
          ('NORTH CENTER', 7541),
          ('NORTH LAWNDALE', 5146),
          ('NORTH PARK', 4210),
          ('NORWOOD PARK', 6469),
          ('OAKLAND', 140),
          ('OHARE', 786),
          ('PORTAGE PARK', 6954),
          ('PULLMAN', 1620),
          ('RIVERDALE', 1547),
          ('ROGERS PARK', 4068),
          ('ROSELAND', 7020),
          ('SOUTH CHICAGO', 4043),
          ('SOUTH DEERING', 1859),
          ('SOUTH LAWNDALE', 14793),
          ('SOUTH SHORE', 4543),
          ('UPTOWN', 4388),
          ('WASHINGTON HEIGHTS', 4006),
          ('WASHINGTON PARK', 2648),
          ('WEST ELSDON', 3700),
          ('WEST ENGLEWOOD', 5946),
          ('WEST GARFIELD PARK', 2622),
          ('WEST LAWN', 4207),
          ('WEST PULLMAN', 3240),
          ('WEST RIDGE', 8197),
          ('WEST TOWN', 9429),
          ('WOODLAWN', 4206)]
\end{Verbatim}
            
    Double-click \textbf{here} for a hint

    Double-click \textbf{here} for another hint

    Double-click \textbf{here} for the solution.

    \subsubsection{Problem 9}\label{problem-9}

\subparagraph{\texorpdfstring{Get the 5 Community Areas with the least
total \textbf{College Enrollment (number of students)} sorted in
ascending
order}{Get the 5 Community Areas with the least total College Enrollment (number of students) sorted in ascending order}}\label{get-the-5-community-areas-with-the-least-total-college-enrollment-number-of-students-sorted-in-ascending-order}

    \begin{Verbatim}[commandchars=\\\{\}]
{\color{incolor}In [{\color{incolor}21}]:} \PY{o}{\PYZpc{}}\PY{k}{sql} select \PYZdq{}Community\PYZus{}Area\PYZus{}Name\PYZdq{}, sum(\PYZdq{}College\PYZus{}Enrollment\PYZus{}\PYZus{}number\PYZus{}of\PYZus{}students\PYZus{}\PYZdq{}) AS TOTAL\PYZus{}ENROLLMENT \PYZbs{}
            \PY{k+kn}{from} \PY{n+nn}{SCHOOLS} \PYZbs{}
            \PY{n}{group} \PY{n}{by} \PY{l+s+s2}{\PYZdq{}}\PY{l+s+s2}{Community\PYZus{}Area\PYZus{}Name}\PY{l+s+s2}{\PYZdq{}} \PYZbs{}
            \PY{n}{order} \PY{n}{by} \PY{n}{TOTAL\PYZus{}ENROLLMENT} \PY{n}{asc} \PYZbs{}
            \PY{n}{fetch} \PY{n}{first} \PY{l+m+mi}{5} \PY{n}{rows} \PY{n}{only}
\end{Verbatim}


    \begin{Verbatim}[commandchars=\\\{\}]
 * ibm\_db\_sa://hcg90289:***@dashdb-txn-sbox-yp-dal09-04.services.dal.bluemix.net:50000/BLUDB
Done.

    \end{Verbatim}

\begin{Verbatim}[commandchars=\\\{\}]
{\color{outcolor}Out[{\color{outcolor}21}]:} [('OAKLAND', 140),
          ('FULLER PARK', 531),
          ('BURNSIDE', 549),
          ('OHARE', 786),
          ('LOOP', 871)]
\end{Verbatim}
            
    Double-click \textbf{here} for a hint

    Double-click \textbf{here} for the solution.

    \subsection{Summary}\label{summary}

\subparagraph{In this lab you learned how to work with a real word
dataset using SQL and Python. You learned how to query columns with
spaces or special characters in their names and with mixed names. You
also used built in database
functions.}\label{in-this-lab-you-learned-how-to-work-with-a-real-word-dataset-using-sql-and-python.-you-learned-how-to-query-columns-with-spaces-or-special-characters-in-their-names-and-with-mixed-names.-you-also-used-built-in-database-functions.}

    Copyright © 2018
\href{cognitiveclass.ai?utm_source=bducopyrightlink\&utm_medium=dswb\&utm_campaign=bdu}{cognitiveclass.ai}.
This notebook and its source code are released under the terms of the
\href{https://bigdatauniversity.com/mit-license/}{MIT License}.


    % Add a bibliography block to the postdoc
    
    
    
    \end{document}
